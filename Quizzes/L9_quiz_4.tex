\documentclass[10pt]{article}

\usepackage[letterpaper,left=0.5in,right=0.5in,top=1in,bottom=1in]{geometry}

\usepackage[T1]{fontenc}
\usepackage[utf8]{inputenc}
\usepackage{lmodern}

\usepackage[activate={true,nocompatibility},final,tracking=true,kerning=true,spacing=true,factor=1100,stretch=10,shrink=10]{microtype}
\microtypecontext{spacing=nonfrench}
\usepackage{xspace}
\usepackage{amssymb,amsfonts,amsmath}
\usepackage{lipsum,xcolor}
\usepackage{graphicx}
\usepackage{float,caption,subcaption,wrapfig}
\usepackage{tcolorbox}
\usepackage{listings}
\usepackage{courier}
\usepackage[english]{babel}
\usepackage{textcomp}
\usepackage{csquotes}
\usepackage{siunitx}
\usepackage[makeroom]{cancel}
\usepackage[shortlabels]{enumitem}
\sisetup{mode=text,
         group-separator={,},
         detect-all,
         binary-units,
         list-units = single,
         range-units = single,
         range-phrase = --,
         per-mode = symbol-or-fraction,
         list-final-separator = {, and }
}

\lstset{basicstyle=\ttfamily,
        breaklines=true,
        numbersep=-8pt,
        numberstyle=\small,
        numbers=right,
        frame = single, 
        showstringspaces=false,    
        keywordstyle=\color{blue}\bf,
        commentstyle=\color{darkgray},
        stringstyle=\color{purple}\bf,
  }

\DeclareSIUnit\atm{atm}
\DeclareSIUnit\bar{bar}

\headheight = 13.6pt
\usepackage{fancyhdr}
\pagestyle{fancy}

\lhead{PH 641 Sp2020}
\chead{Quiz 4}
\rhead{Due 5:00 pm, 18 April 2020}

\rfoot{Submitted by: Paige Lorson}
\tcbset{width=(\linewidth-2mm),before=,after=\hfill,colframe=black,colback=white,}
\newcommand{\volume}{{\ooalign{\hfil$\vol$\hfil\cr\kern0.08em--\hfil\cr}}}
\newenvironment{Solution}
    {\textbf{Solution:}
    
    \vspace{5mm}
    \begin{tcolorbox}
    }
    {
    \end{tcolorbox}
    \vspace{5mm}
    % \newpage
    }
\newcommand{\vol}{{\ooalign{\hfil$V$\hfil\cr\kern0.08em--\hfil\cr}}}
\renewcommand\labelitemi{$\cdot$}

\begin{document}

\noindent\textbf{Quiz problem 1:}  Construct a thermodynamic potential with natural variables entropy $S,$ pressure $p,$ and particle number $N:$ Enthalpy $H(T, p, N)$
\begin{enumerate}
\item  Derive the thermodynamic identity for $H$

\begin{Solution}
Recall that,
\begin{equation}
    dU = TdS - pd\vol +\mu dN 
\end{equation}
where,
\begin{equation}
    \left.\frac{\partial U}{\partial S}\right|_{\scriptsize{\vol}, N} = T \qquad \left.\frac{\partial U}{\partial \vol}\right|_{S, N} = -p \qquad \left.\frac{\partial U}{\partial N}\right|_{\scriptsize{S,\vol}} = \mu  
\end{equation}
We want to transform $U(S,\vol,N)$ to $H(T, p, N)$, so let's define 
\begin{align}
    H&\equiv U - \vol\left.\frac{\partial U}{\partial p}\right|_{S, N}\\
    &\equiv U + p\vol 
\end{align}  
So,
\begin{equation}
    \boxed{
    H = U + p\vol 
    }
\end{equation}
\end{Solution}

\item Using stability considerations for $H$ (Taylor series expansion), show that $C_{p} \geq 0$

\begin{Solution}
\begin{equation}
    dH = TdS + \vol dp +\mu dN
\end{equation}
$C_p$ is defined as the partial of $H$ with respect to $T$ at constant $p$, So
\begin{align}
    C_p &= \left.\frac{\partial H}{\partial T}\right|_p
\end{align}

The Taylor series expansion of $H$ is,
\begin{align}
    H\left(T,p,N\right) \approx H\left(T,p,N_0\right) &+ \left[\left.\frac{\partial H}{\partial T}\right|_{p,N} dT + \left.\frac{\partial H}{\partial p}\right|_{T,N} dp  + \left.\frac{\partial H}{\partial N}\right|_{T,p} dN\right] \\&+ \frac{1}{4}\left[\left.\frac{\partial^2 H}{{\partial T}^2}\right|_{p,N} dT^2 + \left.\frac{\partial^2 H}{{\partial p}^2}\right|_{T,N} dp^2  + \left.\frac{\partial^2 H}{{\partial N}^2}\right|_{T,p} dN^2\right] \\&+ \frac{1}{4}\left[\left.\frac{\partial^2 H}{{\partial T \partial p}}\right|_{N} dT dp + \left.\frac{\partial^2 H}{{\partial p \partial N}}\right|_{T} dp dN  + \left.\frac{\partial^2 H}{{\partial N \partial T}}\right|_{p} dN dT\right]
\end{align}

\begin{align}
    dH\left(T,p,N\right) \approx C_p dT &+ \vol dp  + \mu dN \\+& \frac{1}{4}\left[\left.\frac{\partial^2 H}{{\partial T}^2}\right|_{p,N} dT^2 + \left.\frac{\partial^2 H}{{\partial p}^2}\right|_{T,N} dp^2  + \left.\frac{\partial^2 H}{{\partial N}^2}\right|_{T,p} dN^2\right]\\&+ \frac{1}{4}\left[\left.\frac{\partial^2 H}{{\partial T \partial p}}\right|_{N} dT dp + \left.\frac{\partial^2 H}{{\partial p \partial N}}\right|_{T} dp dN  + \left.\frac{\partial^2 H}{{\partial N \partial T}}\right|_{p} dN dT\right]
\end{align}
The expression in convex, so we get that all of diagonal second order partials are strictly positive. Consider the case where only $dT$ is non-zero. Then,

\begin{align}
    dH\left(T,p,N\right) \approx C_p dT &+  \frac{1}{4}\left.\frac{\partial^2 H}{{\partial T}^2}\right|_{p,N} dT^2 \geq 0
\end{align}
Since the second order can not be negative, neither can $C_p$. Thus,
\begin{equation}
    \boxed{C_p \geq 0}
\end{equation}

% \begin{equation}
%     H\left(T\right)_{p_0,N_0} \approx H(T_0) + \frac{\partial H}{\partial T} dT
% \end{equation}
% So,
% \begin{equation}
%     \frac{\partial H}{\partial T} \approx \frac{H\left(T\right) - H\left(T_0\right)}{dT}
% \end{equation}

\end{Solution}

% \item (Optional, no need to turn in) Calculate all Maxwell relations that can
% be derived from $H$

% \begin{Solution}

% \end{Solution}

\end{enumerate}
\end{document}