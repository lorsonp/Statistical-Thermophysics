\documentclass[10pt]{article}

\usepackage[letterpaper,left=0.5in,right=0.5in,top=1in,bottom=1in]{geometry}

\usepackage[T1]{fontenc}
\usepackage[utf8]{inputenc}
\usepackage{lmodern}

\usepackage[activate={true,nocompatibility},final,tracking=true,kerning=true,spacing=true,factor=1100,stretch=10,shrink=10]{microtype}
\microtypecontext{spacing=nonfrench}
\usepackage{xspace}
\usepackage{amssymb,amsfonts,amsmath}
\usepackage{lipsum,xcolor}
\usepackage{graphicx}
\usepackage{float,caption,subcaption,wrapfig}
\usepackage{tcolorbox}
\usepackage{listings}
\usepackage{courier}
\usepackage[english]{babel}
\usepackage{textcomp}
\usepackage{csquotes}
\usepackage{siunitx}
\usepackage[makeroom]{cancel}
\usepackage[shortlabels]{enumitem}
\sisetup{mode=text,
         group-separator={,},
         detect-all,
         binary-units,
         list-units = single,
         range-units = single,
         range-phrase = --,
         per-mode = symbol-or-fraction,
         list-final-separator = {, and }
}

\lstset{basicstyle=\ttfamily,
        breaklines=true,
        numbersep=-8pt,
        numberstyle=\small,
        numbers=right,
        frame = single, 
        showstringspaces=false,    
        keywordstyle=\color{blue}\bf,
        commentstyle=\color{darkgray},
        stringstyle=\color{purple}\bf,
  }

\DeclareSIUnit\atm{atm}
\DeclareSIUnit\bar{bar}

\headheight = 13.6pt
\usepackage{fancyhdr}
\pagestyle{fancy}

\lhead{PH 641 Sp2020}
\chead{Quiz 3}
\rhead{Due 5:00 pm, 16 April 2020}

\rfoot{Submitted by: Paige Lorson}
\tcbset{width=(\linewidth-2mm),before=,after=\hfill,colframe=black,colback=white,}
\newcommand{\volume}{{\ooalign{\hfil$\vol$\hfil\cr\kern0.08em--\hfil\cr}}}
\newenvironment{Solution}
    {\textbf{Solution:}
    
    \vspace{5mm}
    \begin{tcolorbox}
    }
    {
    \end{tcolorbox}
    \vspace{5mm}
    % \newpage
    }
\newcommand{\vol}{{\ooalign{\hfil$V$\hfil\cr\kern0.08em--\hfil\cr}}}
\renewcommand\labelitemi{$\cdot$}

\begin{document}

\noindent\textbf{Quiz problem 1:} Construct a thermodynamic potential with natural variables temperature $T$ pressure $p,$ and particle number $N:$ Gibbs free energy $G(T, p, N)$
\begin{enumerate}
\item Derive the thermodynamic identity for $G$.

\begin{Solution}
Recall that,
\begin{equation}
    dU = TdS - pd\vol +\mu dN 
\end{equation}
where,
\begin{equation}
    \left.\frac{\partial U}{\partial S}\right|_{\scriptsize{\vol}, N} = T \qquad \left.\frac{\partial U}{\partial \vol}\right|_{S, N} = -p \qquad \left.\frac{\partial U}{\partial N}\right|_{\scriptsize{S,\vol}} = \mu  
\end{equation}
We want to transform $U(S,\vol,N)$ to $G(T, p, N)$, so let's define 
\begin{align}
    G&\equiv U - S\left.\frac{\partial U}{\partial S}\right|_{\scriptsize{\vol}}  + p\left.\frac{\partial U}{\partial \vol}\right|_{S, N}\\
    &\equiv U - ST + p\vol 
\end{align}  
So,
\begin{equation}
    \boxed{
    G = U - ST + p\vol 
    }
\end{equation}
\end{Solution}

\item Calculate all Maxwell relations that can be derived from $G$.

\begin{Solution}
The total differential of $G$ is,
\begin{align}
    dG &= \left.\frac{\partial G}{\partial T}\right|_{p, N} dT + \left.\frac{\partial G}{\partial p}\right|_{T, N} dp + \left.\frac{\partial G}{\partial N}\right|_{T, p} dN\\
    &= \left.\frac{\partial U}{\partial T}\right|_{p, N} dT + \left.\frac{\partial U}{\partial p}\right|_{T, N} dp + \left.\frac{\partial U}{\partial N}\right|_{T, p} dN + \left.\frac{\partial \left[ST + p\vol\right]}{\partial T}\right|_{p, N} dT + \left.\frac{\partial \left[ST + p\vol\right]}{\partial p}\right|_{T, N} dp + \left.\frac{\partial \left[ST + p\vol\right]}{\partial N}\right|_{T, p} dN\\
    &= -S dT +\vol dp +\mu dN
\end{align}

\end{Solution}
\newpage
\item Show that for extensive systems the chemical potential is given by:
\begin{equation}
    \mu=\frac{G}{N}
\end{equation}
\begin{Solution}
Recall the Euler equation, for an extensive system
\begin{equation}
    U = TS - p\vol + \mu N
\end{equation}
We can use this in our definition of $G$,
\begin{equation}
    G = \left[ \bcancel{TS} - \bcancel{p\vol} + \mu N \right] - \bcancel{ST} + \bcancel{p\vol} 
\end{equation}
So for a extensive system,
\begin{equation}
    \boxed{
    \mu = \frac{G}{N}
    }
\end{equation}

\end{Solution}    
    
\end{enumerate}
\end{document}