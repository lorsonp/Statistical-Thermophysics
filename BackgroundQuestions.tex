\documentclass[10pt]{article}

\usepackage[letterpaper,left=0.5in,right=0.5in,top=1in,bottom=1in]{geometry}

\usepackage[T1]{fontenc}
\usepackage[utf8]{inputenc}
\usepackage{lmodern}

\usepackage[activate={true,nocompatibility},final,tracking=true,kerning=true,spacing=true,factor=1100,stretch=10,shrink=10]{microtype}
\microtypecontext{spacing=nonfrench}
\usepackage{xspace}
\usepackage{amssymb,amsfonts,amsmath}
\usepackage{lipsum,xcolor}
\usepackage{graphicx}
\usepackage{float,caption,subcaption,wrapfig}
\usepackage{tcolorbox}
\usepackage{listings}
\usepackage{courier}
\usepackage[english]{babel}
\usepackage{textcomp}
\usepackage{csquotes}
\usepackage{siunitx}
\sisetup{mode=text,
         group-separator={,},
         detect-all,
         binary-units,
         list-units = single,
         range-units = single,
         range-phrase = --,
         per-mode = symbol-or-fraction,
         list-final-separator = {, and }
}

\lstset{basicstyle=\ttfamily,
        breaklines=true,
        numbersep=-8pt,
        numberstyle=\small,
        numbers=right,
        frame = single, 
        showstringspaces=false,    
        keywordstyle=\color{blue}\bf,
        commentstyle=\color{darkgray},
        stringstyle=\color{purple}\bf,
  }

\DeclareSIUnit\atm{atm}
\DeclareSIUnit\bar{bar}

\headheight = 13.6pt
\usepackage{fancyhdr}
\pagestyle{fancy}

\lhead{PH 641 Sp2020}
\chead{Background Questions}
\rhead{Due 1:00 pm, 31 March 2020}

\rfoot{Submitted by: Paige Lorson}
\tcbset{width=(\linewidth-2mm),before=,after=\hfill,colframe=black,colback=white,}
\newcommand{\volume}{{\ooalign{\hfil$V$\hfil\cr\kern0.08em--\hfil\cr}}}
\newenvironment{Solution}
    {\textbf{Solution:}
    
    \vspace{5mm}
    \begin{tcolorbox}
    }
    {
    \end{tcolorbox}
    \vspace{5mm}
    % \newpage
    }


\begin{document}

\noindent\textbf{Note:} This survey serves to inform your instructor and to test the homework submission via gradescope. It will be graded like a regular homework set.

\begin{enumerate}

%%%%%%%%%%%%%%%%%%%%%%%%%%%%%%%%%%%%%%%%%%%%%%%


\item List previous courses in Thermodynamics and Statistical Mechanics? (Please give institution, length (x terms,  x semesters), textbooks)

\begin{Solution}
\textbf{Directly Related:}
\vspace{1mm}

Intro to Thermodynamics\\Penn State University \\1 Semester (3 credits)\\Thermodynamics, An Engineering Approach; Yunus A. Çengel, Michael A. Boles
\vspace{5mm}

Intermediate Thermodynamics\\Oregon State University\\1 term (4 credits)\\Thermodynamics,  1st Ed; S. A. Klein,  G. Nellis
\vspace{5mm}

\textbf{Not so Directly Related:}
\vspace{1mm}

Combustion (Hot Thermo)\\Oregon State University\\1 term (4 credits)\\An Introduction to Combustion, Concepts and Applications;Stephen Turns
\vspace{5mm}

Gas Dynamics (Fast Thermo)\\Oregon State University\\1 term (4 credits)\\Fundamental of Gas Dynamics; Zucker, Biblarz

\end{Solution}
% \newpage

\item How would you rate your background in Thermodynamics? 

\begin{Solution}
I believe I have a pretty strong background in Thermo. I have had quite a few classes on the subject and also passed the Thermodymanics oral and written exams for my department (MIME Thermal-Fluid Sciences).

\end{Solution}
% \newpage 

\item How would you rate your background in Statistical Mechanics? 

\begin{Solution}
I do not have a strong background in statistical mechanics. We touched on the topic briefly in my intermediate thermo class.

\end{Solution}
\newpage 

\item What is your field of interest/research? (For example: solid state (materials, organic, nanotechnology), biophysics, optics, astrophysics, particles, undecided, etc.)

\begin{Solution}
I am a student in the Mechanical Engineering department with a focus in Thermal-Fluid Sciences. My research is on the modeling of soot/nanoparticles for direct numerical simulations (DNS) using a markov chain monte carlo solver on GPUs. 

\end{Solution}
% \newpage 

\item What do you expect to learn in this course (Ph641 and Ph642)?

\begin{Solution}
From these classes, I hope to gain a better understanding of how particles behave at the microscopic level and that this knowledge will help me better understand the physical side of my research.
\vspace{5mm}

\textbf{Side Note:} My preferred form of note taking has always been writing everything by hand and I get a little concerned when a professor uses slides. I completely understand the justification for using them, given the circumstances. However, I was wondering if the pace of the course will still allow for hand writing all notes or if I should adjust my note taking plan. Thanks! 

\end{Solution}
% \newpage
\end{enumerate}
\end{document}
