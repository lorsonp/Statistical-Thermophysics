\documentclass[10pt]{article}

\usepackage[letterpaper,left=0.5in,right=0.5in,top=1in,bottom=1in]{geometry}

\usepackage[T1]{fontenc}
\usepackage[utf8]{inputenc}
\usepackage{lmodern}

\usepackage[activate={true,nocompatibility},final,tracking=true,kerning=true,spacing=true,factor=1100,stretch=10,shrink=10]{microtype}
\microtypecontext{spacing=nonfrench}
\usepackage{xspace}
\usepackage{amssymb,amsfonts,amsmath}
\usepackage{lipsum,xcolor}
\usepackage{graphicx}
\usepackage{float,caption,subcaption,wrapfig}
\usepackage{tcolorbox}
\usepackage{listings}
\usepackage{courier}
\usepackage[english]{babel}
\usepackage{textcomp}
\usepackage{csquotes}
\usepackage{siunitx}
\sisetup{mode=text,
         group-separator={,},
         detect-all,
         binary-units,
         list-units = single,
         range-units = single,
         range-phrase = --,
         per-mode = symbol-or-fraction,
         list-final-separator = {, and }
}

\lstset{basicstyle=\ttfamily,
        breaklines=true,
        numbersep=-8pt,
        numberstyle=\small,
        numbers=right,
        frame = single, 
        showstringspaces=false,    
        keywordstyle=\color{blue}\bf,
        commentstyle=\color{darkgray},
        stringstyle=\color{purple}\bf,
  }

\DeclareSIUnit\atm{atm}
\DeclareSIUnit\bar{bar}

\headheight = 13.6pt
\usepackage{fancyhdr}
\pagestyle{fancy}

\lhead{PH 641 Sp2020}
\chead{Assignment 1}
\rhead{Due 11:59 pm, 17 March 2020}

\rfoot{Submitted by: Paige Lorson}
\tcbset{width=(\linewidth-2mm),before=,after=\hfill,colframe=black,colback=white,}
\newcommand{\volume}{{\ooalign{\hfil$V$\hfil\cr\kern0.08em--\hfil\cr}}}
\newenvironment{Solution}
    {\textbf{Solution:}
    
    \vspace{5mm}
    \begin{tcolorbox}
    }
    {
    \end{tcolorbox}
    \vspace{5mm}
    % \newpage
    }
\newcommand{\vol}{{\ooalign{\hfil$V$\hfil\cr\kern0.08em--\hfil\cr}}}
\renewcommand\labelitemi{$\cdot$}

\begin{document}

% \noindent\textbf{Note:} This survey serves to inform your instructor and to test the homework submission via gradescope. It will be graded like a regular homework set.

\begin{enumerate}

%%%%%%%%%%%%%%%%%%%%%%%%%%%%%%%%%%%%%%%%%%%%%%%

\item Thermodynamic response functions
\begin{enumerate}

\item Derive the relation: $\kappa_{T}-\kappa_{S}=\frac{T V}{C_{p}} \alpha^{2}$
Hint: Do this systematically: for $C_{p}-C_{V}$ we had the entropy $S$ as starting point. Here, no standard thermodynamic potential is an obvious starting point, you really need $V(p) .$ Start with $V(p, X)$ and make educated choices for $X$
\begin{Solution}

\begin{equation}
    \left.\frac{\partial V}{\partial p}\right|_{S,N} =   \left.\frac{\partial V}{\partial p}\right|_{V,N} + 
    \left.\frac{\partial V}{\partial T}\right|_{p,N} 
    \left.\frac{\partial T}{\partial S}\right|_{p,N}
    \left.\frac{\partial S}{\partial P}\right|_{T,N}
\end{equation}
\begin{equation}
    \left.\frac{\partial V}{\partial p}\right|_{S,N} - \left.\frac{\partial V}{\partial p}\right|_{V,N} =     \left.\frac{\partial T}{\partial S}\right|_{p,N} \left.{\frac{\partial V}{\partial T}}^2 \right|_{p,N}
\end{equation}
\begin{equation}
    \left.\frac{\partial V}{\partial p}\right|_{S,N} - \left.\frac{\partial V}{\partial p}\right|_{V,N} = \frac{1}{\left.\frac{\partial S}{\partial T}\right|_{p,N}}     \left.{\frac{\partial V}{\partial T}}^2\right|_{p,N} 
\end{equation}
\begin{equation}
    \frac{1}{V}\left.\frac{\partial V}{\partial p}\right|_{S,N} - \frac{1}{V}\left.\frac{\partial V}{\partial p}\right|_{T,N} = \frac{TV}{T\left.\frac{\partial S}{\partial T}\right|_{p,N}} \frac{1}{V^2}    \left.{\frac{\partial V}{\partial T}}^2\right|_{p,N} 
\end{equation}
\begin{equation}
    \kappa_{T}-\kappa_{S}=\frac{T V}{C_{p}} \alpha^{2}
\end{equation}

\end{Solution}
% \newpage

\item Derive the relation: $\frac{C_{p}}{C_{v}}=\frac{\kappa_{T}}{\kappa_{S}}$
\begin{enumerate}
\item Using the results in part a) and $C_{p}-C_{V}$ derived in class

\begin{Solution}
From class we have that, 
\begin{equation}
    C_p -C_v = \frac{VT \alpha^2}{\kappa_T}
\end{equation}

\begin{equation}
    \frac{\kappa_T}{C_p}\left[C_p -C_v\right] = \left[\frac{VT \alpha^2}{\kappa_T}\right]\frac{\kappa_T}{C_p}
\end{equation}
\begin{equation}
    \kappa_T -C_v\frac{\kappa_T}{C_p} = \left[\frac{VT \alpha^2}{\kappa_T}\right]\frac{\kappa_T}{C_p} = \kappa_{T}-\kappa_{S}
\end{equation}
\begin{equation}
    C_v\frac{\kappa_T}{C_p} = \kappa_{T}-\kappa_{S}\qquad \boxed{\frac{C_{p}}{C_{v}}=\frac{\kappa_{T}}{\kappa_{S}}}
\end{equation}

\end{Solution}

\item Directly without these relations.

\begin{Solution}

\begin{equation}
    {\left.\frac{\partial V}{\partial p}\right|_{S,N}}{\left.\frac{\partial S}{\partial T}\right|_{p,N}} = {\left.\frac{\partial V}{\partial p}\right|_{T,N}}{\left.\frac{\partial S}{\partial T}\right|_{V,N}}
\end{equation}
\begin{equation}
    \frac{T\left.\frac{\partial S}{\partial T}\right|_{p,N}}{T\left.\frac{\partial S}{\partial T}\right|_{V,N}} = \frac{- \frac{1}{V}\left.\frac{\partial V}{\partial p}\right|_{T,N}}{- \frac{1}{V}\left.\frac{\partial V}{\partial p}\right|_{S,N}}
\end{equation}
\end{Solution}
\end{enumerate}

\end{enumerate}
\newpage

\item A one dimensional system (not everything is an ideal gas) For an elastic filament it is found that, at a finite range in temperature, a displacement $x$ requires a force (tension)
$$
\tau=a x-b T+c T x
$$
where $a, b,$ and $c$ are constants. Furthermore, its heat capacity of constant displacement is proportional to temperature, that is $C_{x}=A(x) T$
\begin{enumerate}
    \item Use an appropriate Maxwell relation to calculate $(\partial S / \partial x)_{T}$
    
    \begin{Solution}
    Consider Helmholtz free energy for this system,
    \begin{equation}
        dF = - SdT + \tau dx
    \end{equation}
    we can use Maxwell relations to say,
    \begin{equation}
        \left.\frac{\partial S}{\partial x}\right|_T = -\frac{\partial \tau}{\partial T}_x
    \end{equation}
    so, 
    \begin{equation}
        \boxed{\left.\frac{\partial S}{\partial x}\right|_T = b-cx}
    \end{equation}
    % We can say that ${\partial U}/{\partial x} = \tau$ and we know that, ${\partial U}/{\partial T} = -S$
 
    \end{Solution}
    \item Show that $A$ has to be in fact be independent of $x,$ that is, $\mathrm{d} A / \mathrm{d} x=0$ and hence $A=A_{0}$
    
    \begin{Solution}
    We are given that $C_{x}=A(x) T$. Heat capacity at constant length is also, $C_{x}={\partial U/\partial T}_x $. This means, 
    \begin{equation}
        C_x = T\left.\frac{\partial S}{\partial T}\right|_x = A(x) T
    \end{equation}
    \begin{equation}
        \left.\frac{\partial S}{\partial T}\right|_x = A(x)
    \end{equation}
    
    We can now use this to get $\mathrm{d} A / \mathrm{d} x$.
    \begin{equation}
        \frac{dA}{dx} = \frac{d}{dx}\left[\left.\frac{\partial S}{\partial T}\right|_x\right] = \frac{d}{dx}\left[b-xc\right] 
    \end{equation}
    \begin{equation}
        \boxed{\frac{dA}{dx} = 0 }
    \end{equation}
    \end{Solution}
    \newpage
    \item Derive an expression for $S(T, x)$ assuming $S(T=0, x=0)=S_{0}$
    
    \begin{Solution}
    \begin{align}
        S(T, x) &= \int \left.\frac{\partial S}{\partial T}\right|_x dT + \int \left.\frac{\partial S}{\partial x}\right|_T dx + S_0\\
        &= \int A dT + \int (b-cx) dx + S_0
    \end{align}
    So, 
    \begin{equation}
        \boxed{S(T,x) = AT + bx - \frac{1}{2}c x^2 + S_0}
    \end{equation}
    
    \end{Solution}
    \item Calculate the heat capacity at constant tension, that is,
$$
C_{\tau}=T\left(\frac{\partial S}{\partial T}\right)_{\tau}
$$
as a function of $T$ and $\tau$
    
    \begin{Solution}

    \begin{equation}
    \left.\frac{\partial S}{\partial T}\right|_{\tau}=\left.\frac{\partial S}{\partial T}\right|_{x}+\left.\frac{\partial S}{\partial x}\right|_{T} \left.\frac{\partial x}{\partial T}\right|_{\tau}
    \end{equation}
    
    \begin{equation}
        \left.\frac{\partial S}{\partial T}\right|_{x} = A_0 \qquad \left.\frac{\partial S}{\partial x}\right|_T = b - c x \qquad \left.\frac{\partial x}{\partial T}\right|_{\tau} = \frac{b a -\tau c}{\left(a + c T\right)^2}
    \end{equation}
    
    \begin{equation}
        \left.\frac{\partial S}{\partial T}\right|_{\tau} = A_0 + \left(b - c x\right)\frac{b a -\tau c}{\left(a + c T\right)^2}
    \end{equation}
    \begin{equation}
        \boxed{C_{\tau} = TA_0 + T\left(b - c x\right)\frac{b a -\tau c}{\left(a + c T\right)^2}}
    \end{equation}
    \end{Solution}

\end{enumerate}
\newpage

\item Consider an ideal gas consisting of $N$ particles (molecules) with $f$ degrees of freedom per molecule. The equation of state and the internal energy $U$ for this gas are given by
$$
\begin{array}{l}
p V=N k T \\
U=\frac{f}{2} N k T
\end{array}
$$
$(f=3 \text { for a monoatomic gas, } f=5$ for a diatomic gas, etc.)


\begin{enumerate}
\item Consider an adiabatic process with constant particle number $N$. Starting from equations $1,2$, show that $p V^{\gamma}=\text {const.}$ and $ T V^{\gamma-1}=\text {const.}$ and express $\gamma$ in terms of $f: \gamma(f)$

\begin{Solution}
\begin{equation}
    d\left( p V \right) = d \left( N k T\right)
\end{equation}
\begin{equation}
    V dp + p dV = NkdT
\end{equation}
But also, 
\begin{equation}
    \frac{f}{2}Nk dT = -p dV
\end{equation}
So, 
\begin{equation}
    {-{N k}p dV} = {\frac{f}{2}Nk}[{V dp + p dV}]
\end{equation}
\begin{equation}
-\left({\frac{f}{2}Nk}+{Nk}\right)p dV = \frac{f}{2}Nk V dp    
\end{equation}
\begin{equation}
    \left(\frac{V_2}{V_1}\right)^{\left(\frac{{\frac{f}{2}Nk}+{Nk}}{\frac{f}{2}Nk}\right)} = \frac{p_1}{p_2}
\end{equation}
So, 
\begin{equation}
   \boxed{ pV^{\frac{{\frac{f}{2}Nk}+{Nk}}{\frac{f}{2}Nk}} = \text{const.}}
\end{equation}

Now, $\gamma$ is given by
\begin{equation}
    \boxed{\gamma = {\frac{{f}+{2}}{f}}}
\end{equation}
and 

\begin{equation}
    \frac{f}{2}Nk dT = -\frac{NkT}{V} dV
\end{equation}
\begin{equation}
    \frac{f}{2T} dT = -\frac{1}{V} dV
\end{equation}
\begin{equation}
    \boxed{TV^\frac{2}{f} = TV^{\gamma-1} = \text{const.}}
\end{equation}

\end{Solution}
\newpage

\item Calculate the heat capacities $C_{p}$ and $C_{V}$ for an ideal gas and write $\gamma=\gamma\left(C_{p}, C_{V}\right)$

\begin{Solution}
\begin{equation}
    \boxed{C_v = \frac{f}{2}Nk} \qquad \boxed{C_p = \frac{f}{2}Nk + Nk}
\end{equation}
so,
\begin{equation}
    \boxed{\gamma = \frac{C_p}{C_v}}
\end{equation}
\end{Solution}

\item Consider an (idealized) four stroke engine (Otto motor) working with an ideal gas as a working medium. The (idealized) cycle is given in the figure below:
Determine the efficiency $\eta=\Delta W / \Delta Q$ of this cycle, where $\Delta W$ is the performed work, and $\Delta Q$ is the generated heat during combustion. Express the efficiency in terms of the compression ratio $V_{3} / V_{4}$ of the engine and the degrees of freedom $f$ of the working gas.

\begin{Solution}
For the Otto cycle, efficiency is 
\begin{equation}
    \eta = 1 - \frac{1}{\frac{V_3}{V_4}^{\frac{{f}+{2}}{f}-1}}
\end{equation}

\end{Solution}
\end{enumerate}
\newpage




\item Mystery system The internal energy of a system is given as function of temperature and volume by
$$
U=a V T^{4}
$$
where $a$ is a positive constant.
\begin{enumerate}
\item Show that $S=b V T^{3}$ and find $b$ in terms of $a$

\begin{Solution}
Recall, that 
\begin{equation}
    \left.\frac{\partial U}{\partial T}\right|_V = T \left.\frac{\partial S}{\partial T}\right|_V
\end{equation}
So then, 
\begin{equation}
     \left.\frac{\partial U}{\partial T}\right|_V = 4aVT^3, \qquad \left.\frac{\partial S}{\partial T}\right|_V = 4aVT^2
\end{equation}
\begin{equation}
    \boxed{S = bVT^3, \qquad b = \frac{4a}{3}}
\end{equation}
\end{Solution}

\item Calculate the Helmholtz free energy $F$, the free energy that depends on
$T, V,$ and $N$

\begin{Solution}
\begin{equation}
    F = U - TS
\end{equation}
\begin{equation}
    F = aVT^4 - \frac{4}{3}aVT^4
\end{equation}
\begin{equation}
    \boxed{F = -\frac{1}{3}aVT^4}
\end{equation}
\end{Solution}

\item Calculate the pressure for the system.

\begin{Solution}
\begin{equation}
    p = -\left.\frac{\partial F}{\partial V}\right|_T = \boxed{\frac{1}{3}aT^4}
\end{equation}
\end{Solution}

\item Calculate the chemical potential for the system. (Bonus) What kind of particle can have such a value for its chemical potential?

\begin{Solution}
\begin{equation}
    \mu = G = U - T S + p V
\end{equation}
\begin{equation}
    \boxed{\mu = -\frac{1}{3}aVT^4 + \frac{1}{3}aVT^4 = 0}
\end{equation}
\end{Solution}
% \newpage
    
\end{enumerate}
\end{enumerate}
\end{document}
