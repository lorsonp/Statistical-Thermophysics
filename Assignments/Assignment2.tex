\documentclass[10pt]{article}

\usepackage[letterpaper,left=0.5in,right=0.5in,top=1in,bottom=1in]{geometry}

\usepackage[T1]{fontenc}
\usepackage[utf8]{inputenc}
\usepackage{lmodern}

\usepackage[activate={true,nocompatibility},final,tracking=true,kerning=true,spacing=true,factor=1100,stretch=10,shrink=10]{microtype}
\microtypecontext{spacing=nonfrench}
\usepackage{xspace}
\usepackage{amssymb,amsfonts,amsmath}
\usepackage{lipsum,xcolor}
\usepackage{graphicx}
\usepackage{float,caption,subcaption,wrapfig}
\usepackage{tcolorbox}
\usepackage{listings}
\usepackage{courier}
\usepackage[english]{babel}
\usepackage{textcomp}
\usepackage{csquotes}
\usepackage{siunitx}
\sisetup{mode=text,
         group-separator={,},
         detect-all,
         binary-units,
         list-units = single,
         range-units = single,
         range-phrase = --,
         per-mode = symbol-or-fraction,
         list-final-separator = {, and }
}

\lstset{basicstyle=\ttfamily,
        breaklines=true,
        numbersep=-8pt,
        numberstyle=\small,
        numbers=right,
        frame = single, 
        showstringspaces=false,    
        keywordstyle=\color{blue}\bf,
        commentstyle=\color{darkgray},
        stringstyle=\color{purple}\bf,
  }

\DeclareSIUnit\atm{atm}
\DeclareSIUnit\bar{bar}

\headheight = 13.6pt
\usepackage{fancyhdr}
\pagestyle{fancy}

\lhead{PH 641 Sp2020}
\chead{Assignment 1}
\rhead{Due 11:59 pm, 17 March 2020}

\rfoot{Submitted by: Paige Lorson}
\tcbset{width=(\linewidth-2mm),before=,after=\hfill,colframe=black,colback=white,}
\newcommand{\volume}{{\ooalign{\hfil$V$\hfil\cr\kern0.08em--\hfil\cr}}}
\newenvironment{Solution}
    {\textbf{Solution:}
    
    \vspace{5mm}
    \begin{tcolorbox}
    }
    {
    \end{tcolorbox}
    \vspace{5mm}
    % \newpage
    }
\newcommand{\vol}{{\ooalign{\hfil$V$\hfil\cr\kern0.08em--\hfil\cr}}}
\renewcommand\labelitemi{$\cdot$}

\begin{document}

% \noindent\textbf{Note:} This survey serves to inform your instructor and to test the homework submission via gradescope. It will be graded like a regular homework set.

\begin{enumerate}

%%%%%%%%%%%%%%%%%%%%%%%%%%%%%%%%%%%%%%%%%%%%%%%

\item Thermodynamic response functions
\begin{enumerate}

\item Derive the relation
$$
\kappa_{T}-\kappa_{S}=\frac{T V}{C_{p}} \alpha^{2}
$$
Hint: Do this systematically: for $C_{p}-C_{V}$ we had the entropy $S$ as starting point. Here, no standard thermodynamic potential is an obvious starting point, you really need $V(p) .$ Start with $V(p, X)$ and make educated choices for $X$
\begin{Solution}


\end{Solution}
% \newpage

\item Derive the relation:
$$
\frac{C_{p}}{C_{v}}=\frac{\kappa_{T}}{\kappa_{S}}
$$
\begin{enumerate}
\item Using the results in part a) and $C_{p}-C_{V}$ derived in class

\begin{Solution}

\end{Solution}

\item Directly without these relations.

\begin{Solution}

\end{Solution}
\end{enumerate}

\end{enumerate}
\newpage

\item A one dimensional system (not everything is an ideal gas) For an elastic filament it is found that, at a finite range in temperature, a displacement $x$ requires a force (tension)
$$
\tau=a x-b T+c T x
$$
where $a, b,$ and $c$ are constants. Furthermore, its heat capacity of constant displacement is proportional to temperature, that is $C_{x}=A(x) T$
\begin{enumerate}
    \item Use an appropriate Maxwell relation to calculate $(\partial S / \partial x)_{T}$
    
    \begin{Solution}


    \end{Solution}
    \item Show that $A$ has to be in fact be independent of $x,$ that is, $\mathrm{d} A / \mathrm{d} x=0$ and hence $A=A_{0}$
    
    \begin{Solution}


    \end{Solution}
    \item Derive an expression for $S(T, x)$ assuming $S(T=0, x=0)=S_{0}$
    
    \begin{Solution}


    \end{Solution}
    \item Calculate the heat capacity at constant tension, that is,
$$
C_{\tau}=T\left(\frac{\partial S}{\partial T}\right)_{\tau}
$$
as a function of $T$ and $\tau$
    
    \begin{Solution}


    \end{Solution}

\end{enumerate}
\newpage

\item Consider an ideal gas consisting of $N$ particles (molecules) with $f$ degrees of freedom per molecule. The equation of state and the internal energy $U$ for this gas are given by
$$
\begin{array}{l}
p V=N k T \\
U=\frac{f}{2} N k T
\end{array}
$$
$(f=3 \text { for a monoatomic gas, } f=5$ for a diatomic gas, etc.)


a) 
b) 
c) 
\begin{enumerate}
\item Consider an adiabatic process with constant particle number $N$. Starting from equations $1,2,$ show that
$$
p V^{\gamma}=\text {const.} \quad \text { and } \quad T V^{\gamma-1}=\text {const.}
$$
(3)
and express $\gamma$ in terms of $f: \gamma(f)$
\begin{Solution}


\end{Solution}
% \newpage

\item Calculate the heat capacities $C_{p}$ and $C_{V}$ for an ideal gas and write $\gamma=\gamma\left(C_{p}, C_{V}\right)$
\begin{Solution}

\end{Solution}

\item Consider an (idealized) four stroke engine (Otto motor) working with an ideal gas as a working medium. The (idealized) cycle is given in the figure below:
Determine the efficiency $\eta=\Delta W / \Delta Q$ of this cycle, where $\Delta W$ is the performed work, and $\Delta Q$ is the generated heat during combustion. Express the efficiency in terms of the compression ratio $V_{3} / V_{4}$ of the engine and the degrees of freedom $f$ of the working gas.
\begin{Solution}

\end{Solution}
\end{enumerate}
\newpage




\item Mystery system The internal energy of a system is given as function of temperature and volume by
$$
U=a V T^{4}
$$
where $a$ is a positive constant.
\begin{enumerate}
\item Show that $S=b V T^{3}$ and find $b$ in terms of $a$

\begin{Solution}

\end{Solution}

\item Calculate the Helmholtz free energy $F$, the free energy that depends on
$T, V,$ and $N$

\begin{Solution}

\end{Solution}

\item Calculate the pressure for the system.

\begin{Solution}

\end{Solution}

\item Calculate the chemical potential for the system. (Bonus) What kind of particle can have such a value for its chemical potential?

\begin{Solution}

\end{Solution}
% \newpage
    
\end{enumerate}
\end{enumerate}
\end{document}
